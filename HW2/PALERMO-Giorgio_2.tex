\documentclass[a4paper]{article}

%% Language and font encodings
\usepackage[english,italian]{babel}
\usepackage[utf8x]{inputenc}
\usepackage[T1]{fontenc}

%% Sets page size and margins
\usepackage[a4paper,top=3cm,bottom=2cm,left=3cm,right=3cm,marginparwidth=1.75cm]{geometry}

%% Useful packages
\usepackage{amsmath}
\usepackage{amsfonts}
\usepackage{bm}
\usepackage{graphicx}
\usepackage[colorinlistoftodos]{todonotes}
\usepackage[colorlinks=true, all colors=blue]{hyperref} %referenze linkate
\usepackage{booktabs}
\usepackage{siunitx}  %notaz. espon. con \num{} e unità di misura in SI con \si{}
\usepackage{xcolor}
\usepackage{colortbl}
\usepackage{bm}
\usepackage{caption} 
\usepackage{indentfirst}
\usepackage{physics} 
\usepackage{rotating}
\usepackage{tabularx}
\usepackage{url}
\usepackage{pst-plot}
\usepackage{comment} %per usare l'ambiente {comment}
\usepackage{float} 
\usepackage{subfig}
\usepackage[americanvoltages]{circuitikz} %per disegnare circuiti
\usepackage{tikz}
\usepackage{mathtools} %per allineare su più linee in ambiente {align} o {align*}
\usepackage{cancel}
\renewcommand{\CancelColor}{\color{lightgray}}
%\setlength{\parindent}{0cm}


\graphicspath{{Figure/}}
\captionsetup{format=hang,labelfont={sf,bf},font=small}
\captionsetup{tableposition=top,figureposition=bottom,font=small}
\captionsetup[table]{skip=8pt}
%Comando per l'unit\'a di misura A^1\2
\newcommand{\radamp}[0]{\ \text{A}^{1/2}}
\newcommand{\kz}{K^0}
\newcommand{\bkz}{\bar{K}^0}
\newcommand{\kzs}{\ket{K^0}}
\newcommand{\bkzs}{\ket{\bar{K}^0}}
%\renewcommand{\dag}[1]{{#1}^\dagger}


\title{Domanda SubNuc}
\author{Giorgio Palermo}




\begin{document}
\hypersetup{linkcolor = black}
\hypersetup{linkcolor = blue}

\begin{center}
    \textbf{MASTER'S DEGREE IN PHYSICS}
    
    Academic Year 2019-2020
    
    \medskip
    \textbf{Introduction to Many Body Theory}
\end{center}

\vspace{0.8cm}
Student: Giorgio Palermo

Student ID: 1238258

Date: April 28, 2019

\bigskip

\begin{center}
\textbf{HOMEWORK 2}
\end{center}

\section*{First Exercise}
\noindent To compute the Green function for a non interacting system we must start from the general expression for the green function of an homogeneous system:
\begin{align}
G_{\alpha\beta}(\vb{k},\omega) = \hbar V \bqty{\frac{\mel{\phi_0}{\hat{\psi}_\alpha(0)}{n,\vb{k}}\mel{n,\vb{k}}{\hat{\psi}^\dag_\beta(0)}{\phi_0}}{\hbar \omega -\mu -\mathcal{E}_{\vb{k}}^{(N+1)}+i\eta} + 
\frac{\mel{\phi_0}{\hat{\psi}^\dag_\beta(0)}{n,-\vb{k}}\mel{n,-\vb{k}}{\hat{\psi}_\alpha(0)}{\phi_0}}{\hbar \omega -\mu - \mathcal{E}_{\vb{k}}^{(N-1)} -i\eta}}
\label{eq:greenTotal}
\end{align}
where the fields are defined as follows:
\begin{align}
&\hat{\psi}_\alpha(0) = \sum_{\vb{k}'} \varphi_{\vb{k}'}(0)\hat{c}_{\vb{k}',\alpha} 
&\hat{\psi}^\dag_\beta(0)= \sum_{\vb{k}''} \varphi_{\vb{k}''}(0)\hat{c}^\dag_{\vb{k}',\alpha}.
\end{align}
We notice that for a non interacting system a complete set of eigenstates for the momentum operator $\hat{\vec{P}}$ is also a complete set of eigenstates for the Hamiltonian $\hat{H};$ the excited states of the system are identified just by the $\vb{k}$ index and thus the sum over $n$ in \eqref{eq:greenTotal} is unnecessary and can be omitted.

\noindent Let's proceed computing the matrix elements in \eqref{eq:greenTotal}.
The first semplification of the expression is given by the dyad $\dyad{\vb{k}}{\vb{k}},$ which "selects" in the summations over $\vb{k}'$ and $\vb{k}''$ present in the definitions of the fields only the terms with momentum equal to $\vb{k}:$
\begin{align}
\mel{\phi_0}{\hat{\psi}_\alpha(0)}{\vb{k}}\mel{\vb{k}}{\hat{\psi}^\dag_\beta(0)}{\phi_0} &= \sum_{\vb{k}'\vb{k}''}\mel{\phi_0}{\varphi_{\vb{k}'}(0)\hat{c}_{\vb{k}',\alpha}}{\vb{k}} \mel{\vb{k}}{\varphi^\dag_{\vb{k}''}(0)\hat{c}^\dag_{\vb{k}'',\beta}}{\phi_0}\Theta(k - k_F) \label{eq:mel_first_step}\\
&=\sum_{\vb{k}',\vb{k}''} \varphi_{\vb{k}'}(0)\varphi^\dag_{\vb{k}''}(0)\braket{\vb{k}',\alpha}{\vb{k}}\braket{\vb{k}}{\vb{k}'',\beta}\Theta(k - k_F)\\
&=\sum_{\vb{k}',\vb{k}''} \varphi_{\vb{k}'}(0)\varphi^\dag_{\vb{k}''}(0) \delta_{\alpha,\beta}\delta_{\vb{k}',\vb{k}}\delta_{\vb{k}'',\vb{k}}\Theta(k - k_F)\\
&=\abs{\varphi_{\vb{k}}(0)}^2 \delta_{\alpha,\beta}\Theta(k-k_F).
\end{align}
In \eqref{eq:mel_first_step} we introduced the $\Theta(k - k_F):$ this is due to the fact that the field operators are acting in both the matrix elements as creation operators over a filled Fermi sphere: in this situation all contributions to the sum like $\hat{c}^\dag_{\vb{k}}\ket{\vb{k}}$ for $k <k_F$ are annihilated due to the exclusion principle.
The second couple of matrix elements in \eqref{eq:greenTotal} is computed similarly:
\begin{equation}
\mel{\phi_0}{\hat{\psi}^\dag_\beta(0)}{n,-\vb{k}}\mel{n,-\vb{k}}{\hat{\psi}_\alpha(0)}{\phi_0} = \abs{\varphi_{-\vb{k}}(0)}^2\delta_{\alpha,\beta} \Theta(k_F - k);
\end{equation}
in this case the limitation to the sum is applied to all the values of $k$ above the Fermi level, since the ladder operators act in both the matrix element as destruction operators on the ground state $\ket{\phi_0}.$

The denominators both contain an addend which is the excitation energy of the system with $N+1$ particles ($N-1$ respectively).
This term can be rewtitten as a function of the momentum $k,$ since in a non interacting system it indicizes the excited states:
\begin{align}
\mathcal{E}_{\vb{k}}^{(N+1)} &= E_{\vb{k}}^{(N+1)} - E^{(N+1)}\\
&= E^{(N+1)}_{\vb{k}} - E^{(N)} - \pqty{E^{(N+1)} - E^{(N)}}\\
&= \mathcal{E}_{\vb{k}}^0 - \mathcal{E}_{F}^0\\
&= \frac{\hbar^2}{2m} \pqty{k^2 - k_F^2}
\end{align}
and similarly for the $\mathcal{E}_{\vb{k}}^{(N-1)}$ term:
\begin{equation}
\mathcal{E}_{-\vb{k}}^{(N-1)} =  \frac{\hbar^2}{2m} \pqty{k_F^2 - k^2}.
\end{equation}
The chemical potential $\mu$ is assumed equal for the $N \rightarrow N+1$ and $N \rightarrow N-1$ excitations:
\begin{equation}
\mu = E^{(N+1)} - E^{(N)}=E^{(N-1)} - E^{(N)} =\frac{\hbar^2k_F^2}{2m}
\end{equation}




\newpage
\section{Second Exercise}


\end{document}


































