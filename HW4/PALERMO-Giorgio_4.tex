\documentclass[a4paper]{article}

%% Language and font encodings
\usepackage[english,italian]{babel}
\usepackage[utf8x]{inputenc}
\usepackage[T1]{fontenc}

%% Sets page size and margins
\usepackage[a4paper,top=3cm,bottom=2cm,left=2.7cm,right=2.7cm,marginparwidth=1.75cm]{geometry}

%% Useful packages
\usepackage{amsmath}
\usepackage{amsfonts}
\usepackage{bm}
\usepackage{graphicx}
\usepackage[colorinlistoftodos]{todonotes}
\usepackage[colorlinks=true, all colors=blue]{hyperref} %referenze linkate
\usepackage{booktabs}
\usepackage{siunitx}  %notaz. espon. con \num{} e unità di misura in SI con \si{}
\usepackage{xcolor}
\usepackage{colortbl}
\usepackage{bm}
\usepackage{caption} 
\usepackage{indentfirst}
\usepackage{physics} 
\usepackage{rotating}
\usepackage{tabularx}
\usepackage{url}
\usepackage{pst-plot}
\usepackage{comment} %per usare l'ambiente {comment}
\usepackage{float} 
\usepackage{subfig}
\usepackage[americanvoltages]{circuitikz} %per disegnare circuiti
\usepackage{tikz}
\usepackage{mathtools} %per allineare su più linee in ambiente {align} o {align*}
\usepackage{cancel}
\renewcommand{\CancelColor}{\color{lightgray}}
%\setlength{\parindent}{0cm}


\graphicspath{{Figure/}}
\captionsetup{format=hang,labelfont={sf,bf},font=small}
\captionsetup{tableposition=top,figureposition=bottom,font=small}
\captionsetup[table]{skip=8pt}

\newcommand{\meanv}[1]{\langle #1 \rangle}
\renewcommand{\hat}{\widehat}
\newcommand{\anto}{\hat{\psi}_{\lambda'}(\va{x}_1\tau_1)}
\newcommand{\basi}{ \hat{\psi}_{\mu'}(\va{x}_2\tau_2)}
\newcommand{\ciccio}{\hat{\psi}_\mu(\va{x}_2'\tau_2')}
\newcommand{\duccio}{\hat{\psi}_\lambda(\va{x}_1'\tau_1')}
\newcommand{\stanis}{\frac{e^{-\beta\hat{K}}}{Z_G}}
\newcommand{\lisa}{\lim_{\substack{\tau_1'\rightarrow\tau_1^+\\\tau_2'\rightarrow\tau_2^+}}}

\title{Domanda SubNuc}
\author{Giorgio Palermo}




\begin{document}
\hypersetup{linkcolor = black}
\hypersetup{linkcolor = blue}

\begin{center}
    \textbf{MASTER'S DEGREE IN PHYSICS}
    
    Academic Year 2019-2020
    
    \medskip
    \textbf{Introduction to Many Body Theory}
\end{center}

\vspace{0.8cm}
Student: Giorgio Palermo

Student ID: 1238258

Date: June 15, 2020

\bigskip

\begin{center}
\textbf{HOMEWORK 4}
\end{center}



\section*{Exercise 7.1}
For a generic one particle operator $\hat{O}$ the grand canonical ensemble average is defined as
\begin{equation}
\meanv{\hat{O}} = \Tr{\hat{\rho}_G \hat{O}} = \int \dd[3]{\va{x}} \lim_{\va{x}'\rightarrow\va{x}}\Tr{\widehat{\rho}_{G} \hat{\psi}_{\alpha}^{\dagger}\left(\va{x}^{\prime}\right) O_{\alpha \beta}(\va{x}) \hat{\psi}_{\beta}(\va{x})}
= \int \dd[3]{\va{x}} \lim_{\va{x}'\rightarrow\va{x}} O_{\alpha\beta}\Tr{\frac{e^{-\beta\hat{K}}}{Z_G}\hat{\psi}_{\alpha}^{\dagger}\left(\va{x}^{\prime}\right)  \hat{\psi}_{\beta}(\va{x}) }
\end{equation}
where $O_{\alpha\beta}(\va{x})$ is the first quantized form of the operator $\hat{O},$ $\hat{\rho}_G = \exp{-\beta\hat{K}}/Z_G$ is the statistic operator and $\hat{K} = \hat{H} - \mu\hat{N}$ is the grand canonical Hamiltonian.
In analogy it is possible to define an ensemble average for a two particle operator in second quantization which, in the specific case of the interparticle potential $\hat{V}(\va{x} - \va{x}')$ is written as:
\begin{equation}
\meanv{\hat{V}} = \Tr{\hat{\rho}_G\hat{V}} = \int\dd[3]{\va{x}_1}\dd[3]{\va{x}_2} \lim_{\substack{\va{x}_1'\rightarrow\va{x}_1\\\va{x}_2'\rightarrow\va{x}_2}} 
V_{\substack{\mu\mu'\\\lambda\lambda'}}(\va{x}_1 - \va{x}_2)\Tr{\frac{e^{-\beta\hat{K}}}{Z_G}\hat{\psi}^\dag_{\lambda'}(\va{x}_1')\hat{\psi}^\dag_{\mu'}(\va{x}_2')\hat{\psi}_\mu(\va{x}_2)\hat{\psi}_\lambda(\va{x}_1)}
\label{eq:mean_potential}
\end{equation}
Now, it is possible to connect the previous definition to the one of the two particle temperature Green function:
\begin{equation}
\mathcal{G}_{\alpha\beta;\gamma\delta}(\va{x}_1\tau_1,\va{x}_2\tau_2;\va{x}_1'\tau_1',\va{x_2}'\tau_2')
= \Tr{\hat{\rho}_G T\bqty{\hat{\psi}_\alpha(\va{x}_1\tau_1) \hat{\psi}_\beta(\va{x}_2\tau_2);\hat{\psi}_\gamma(\va{x}_2'\tau_2') \hat{\psi}_\delta(\va{x}_1'\tau_1')}}
\label{eq:temp_GF}
\end{equation}
by proving that the trace in expression \eqref{eq:mean_potential} is equal to 
\begin{equation}
\lim_{\substack{\tau_1'\rightarrow\tau_1^+\\\tau_2'\rightarrow\tau_2^+}}
\Tr{\frac{e^{-\beta\hat{K}}}{Z_G}\hat{\psi}_{\lambda'}(\va{x}_1\tau_1) \hat{\psi}_{\mu'}(\va{x}_2\tau_2);\hat{\psi}_\mu(\va{x}_2'\tau_2') \hat{\psi}_\lambda(\va{x}_1'\tau_1')}
\end{equation}
where the dime dependence of the fields is given by the modified Heisenberg picture
\begin{equation}
\hat{\psi}_{\lambda}(\va{x}\tau) = e^{\hat{K}\tau/\hbar}\hat{\psi}_{\lambda}(\va{x})e^{-\hat{K}\tau/\hbar}
\end{equation}
in which we denote the grand canonical Hamiltonian with $\hat{K}$ as before and the time with $\tau.$
To prove this statement we write the explicit dependence on the time of the fields:
\begin{equation}
\lisa\Tr{\stanis e^{\hat{K}\tau_1'/\hbar}A e^{-\hat{K}\tau_1'/\hbar}e^{\hat{K}\tau_2'/\hbar}B \cancel{e^{-\hat{K}\tau_2'/\hbar}e^{\hat{K}\tau_2/\hbar}} C e^{-\hat{K}\tau_2/\hbar} e^{\hat{K}\tau_1/\hbar}D e^{-\hat{K}\tau_1/\hbar}}
\end{equation}
where we denoted the space-dependent fields with the letters $A,B,C,D$ for brevity.
We see that in the limit $\tau_1'\rightarrow\tau_1; \tau_2'\rightarrow\tau_2$ the highlighted factor cancels out.
Using the cyclic property of the trace and the fact that the statistic operator and the time evolution operator commute, we can cancel out another time evolution operator product:
\begin{equation}
\lisa\Tr{\stanis \cancel{e^{-\hat{K}\tau_1/\hbar} e^{\hat{K}\tau_1'/\hbar}}A e^{-\hat{K}\tau_1'/\hbar}e^{\hat{K}\tau_2'/\hbar}B  C e^{-\hat{K}\tau_2/\hbar} e^{\hat{K}\tau_1/\hbar}D }
\end{equation}
The next step is to write down the explicit expression of the trace as a function of a complete set of eigenstates of the grand canonical Hamiltonian such that $\hat{K}\ket{Nj} = E_{Nj}\ket{Nj};$ using again the cyclic property of the trace:
\begin{align}
&\lisa \sum_{Nj} \ev{D \stanis A e^{-\hat{K}\tau_1'/\hbar}e^{\hat{K}\tau_2'/\hbar}B  C e^{-\hat{K}\tau_2/\hbar} e^{\hat{K}\tau_1/\hbar}}{Nj}\\
=& \lisa \sum_{Nj} e^{E_{Nj}\tau_1/\hbar} e^{-E_{Nj}\tau_2/\hbar} \ev{D \stanis A e^{-\hat{K}\tau_1'/\hbar}e^{\hat{K}\tau_2'/\hbar}B  C }{Nj}\\
=& \lisa \sum_{Nj} \cancel{e^{-E_{Nj}\tau_1'/\hbar} e^{E_{Nj}\tau_1/\hbar}}\cancel{e^{-E_{Nj}\tau_2/\hbar} e^{E_{Nj}\tau_2'/\hbar}}\ev{\stanis A B  C  D}{Nj}
\end{align}
which we can rewrite in the same form as present in \eqref{eq:mean_potential}:
\begin{equation}
=\Tr{\frac{e^{-\beta\hat{K}}}{Z_G}\hat{\psi}^\dag_{\lambda'}(\va{x}_1')\hat{\psi}^\dag_{\mu'}(\va{x}_2')\hat{\psi}_\mu(\va{x}_2)\hat{\psi}_\lambda(\va{x}_1)}
\end{equation}
\begin{equation}
\qq{¿TIME ORDERED PRODUCT?}
\end{equation}
Therefore, using the definition of the temperature GF \eqref{eq:temp_GF} we can state that the ensemble average of the interparticle potential is:
\begin{equation}
\meanv{\hat{V}} = \Tr{\hat{\rho}_G\hat{V}} = \int\dd[3]{\va{x}_1}\dd[3]{\va{x}_2} V_{\substack{\mu\mu'\\\lambda\lambda'}}(\va{x}_1 - \va{x}_2)
\mathcal{G}_{\lambda\lambda';\mu\mu'}(\va{x}_1\tau_1,\va{x}_2\tau_2;\va{x}_1\tau_1^+,\va{x}_2\tau_2^+).
\end{equation}






\end{document}


































